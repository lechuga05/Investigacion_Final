%% BioMed_Central_Tex_Template_v1.06
%%                                      %
%  bmc_article.tex            ver: 1.06 %
%                                       %

%%IMPORTANT: do not delete the first line of this template
%%It must be present to enable the BMC Submission system to
%%recognise this template!!

%%%%%%%%%%%%%%%%%%%%%%%%%%%%%%%%%%%%%%%%%
%%                                     %%
%%  LaTeX template for BioMed Central  %%
%%     journal article submissions     %%
%%                                     %%
%%          <8 June 2012>              %%
%%                                     %%
%%                                     %%
%%%%%%%%%%%%%%%%%%%%%%%%%%%%%%%%%%%%%%%%%


%%%%%%%%%%%%%%%%%%%%%%%%%%%%%%%%%%%%%%%%%%%%%%%%%%%%%%%%%%%%%%%%%%%%%
%%                                                                 %%
%% For instructions on how to fill out this Tex template           %%
%% document please refer to Readme.html and the instructions for   %%
%% authors page on the biomed central website                      %%
%% http://www.biomedcentral.com/info/authors/                      %%
%%                                                                 %%
%% Please do not use \input{...} to include other tex files.       %%
%% Submit your LaTeX manuscript as one .tex document.              %%
%%                                                                 %%
%% All additional figures and files should be attached             %%
%% separately and not embedded in the \TeX\ document itself.       %%
%%                                                                 %%
%% BioMed Central currently use the MikTex distribution of         %%
%% TeX for Windows) of TeX and LaTeX.  This is available from      %%
%% http://www.miktex.org                                           %%
%%                                                                 %%
%%%%%%%%%%%%%%%%%%%%%%%%%%%%%%%%%%%%%%%%%%%%%%%%%%%%%%%%%%%%%%%%%%%%%

%%% additional documentclass options:
%  [doublespacing]
%  [linenumbers]   - put the line numbers on margins

%%% loading packages, author definitions

%\documentclass[twocolumn]{bmcart}% uncomment this for twocolumn layout and comment line below
\documentclass{bmcart}

%%% Load packages
%\usepackage{amsthm,amsmath}
%\RequirePackage{natbib}
%\RequirePackage[authoryear]{natbib}% uncomment this for author-year bibliography
%\RequirePackage{hyperref}
\usepackage[utf8]{inputenc} %unicode support
%\usepackage[applemac]{inputenc} %applemac support if unicode package fails
%\usepackage[latin1]{inputenc} %UNIX support if unicode package fails


%%%%%%%%%%%%%%%%%%%%%%%%%%%%%%%%%%%%%%%%%%%%%%%%%
%%                                             %%
%%  If you wish to display your graphics for   %%
%%  your own use using includegraphic or       %%
%%  includegraphics, then comment out the      %%
%%  following two lines of code.               %%
%%  NB: These line *must* be included when     %%
%%  submitting to BMC.                         %%
%%  All figure files must be submitted as      %%
%%  separate graphics through the BMC          %%
%%  submission process, not included in the    %%
%%  submitted article.                         %%
%%                                             %%
%%%%%%%%%%%%%%%%%%%%%%%%%%%%%%%%%%%%%%%%%%%%%%%%%


\def\includegraphic{}
\def\includegraphics{}



%%% Put your definitions there:
\startlocaldefs
\endlocaldefs


%%% Begin ...
\begin{document}

%%% Start of article front matter
\begin{frontmatter}

\begin{fmbox}
\dochead{Investigación}

%%%%%%%%%%%%%%%%%%%%%%%%%%%%%%%%%%%%%%%%%%%%%%
%%                                          %%
%% Enter the title of your article here     %%
%%                                          %%
%%%%%%%%%%%%%%%%%%%%%%%%%%%%%%%%%%%%%%%%%%%%%%

\title{El transporte público.}

%%%%%%%%%%%%%%%%%%%%%%%%%%%%%%%%%%%%%%%%%%%%%%
%%                                          %%
%% Enter the authors here                   %%
%%                                          %%
%% Specify information, if available,       %%
%% in the form:                             %%
%%   <key>={<id1>,<id2>}                    %%
%%   <key>=                                 %%
%% Comment or delete the keys which are     %%
%% not used. Repeat \author command as much %%
%% as required.                             %%
%%                                          %%
%%%%%%%%%%%%%%%%%%%%%%%%%%%%%%%%%%%%%%%%%%%%%%

\author[
   addressref={aff1},                   % id's of addresses, e.g. {aff1,aff2}
   corref={aff1},                       % id of corresponding address, if any
   noteref={n1},                        % id's of article notes, if any
   email={srlechugas@gmail.com}   % email address
]{\inits{JE}\fnm{Carlos Abraham} \snm{Corrales Moreno}}

%%%%%%%%%%%%%%%%%%%%%%%%%%%%%%%%%%%%%%%%%%%%%%
%%                                          %%
%% Enter the authors' addresses here        %%
%%                                          %%
%% Repeat \address commands as much as      %%
%% required.                                %%
%%                                          %%
%%%%%%%%%%%%%%%%%%%%%%%%%%%%%%%%%%%%%%%%%%%%%%

\address[id=aff1]{%                           % unique id
  \orgname{Instituto Tecnologico de Tijuana}, % university, etc
  \street{Av. Tecnologico},                     %
  %\postcode{}                                % post or zip code
  \city{Tijuana, B.C.},                              % city
  \cny{México}                                    % country
}


%%%%%%%%%%%%%%%%%%%%%%%%%%%%%%%%%%%%%%%%%%%%%%
%%                                          %%
%% Enter short notes here                   %%
%%                                          %%
%% Short notes will be after addresses      %%
%% on first page.                           %%
%%                                          %%
%%%%%%%%%%%%%%%%%%%%%%%%%%%%%%%%%%%%%%%%%%%%%%



\end{fmbox}% comment this for two column layout

%%%%%%%%%%%%%%%%%%%%%%%%%%%%%%%%%%%%%%%%%%%%%%
%%                                          %%
%% The Abstract begins here                 %%
%%                                          %%
%% Please refer to the Instructions for     %%
%% authors on http://www.biomedcentral.com  %%
%% and include the section headings         %%
%% accordingly for your article type.       %%
%%                                          %%
%%%%%%%%%%%%%%%%%%%%%%%%%%%%%%%%%%%%%%%%%%%%%%

\begin{abstractbox}

\begin{abstract} % abstract
\parttitle{Titulo} %if any
El transporte público: en Tijuana.

\parttitle{Resumen} %if any 
Esta investigación trata busca encontrar el concepto de lo que es el trasporte público, para así saber si el concepto se está aplicando correctamente en la ciudad. Se busca también una comparativa entre ciudades, para saber las deficiencias e ineficiencias entre ambas ciudades, y poder formular una conclusión sobre lo que sería mejor teniendo en cuenta estas deficiencias e ineficiencias.
\end{abstract}

%%%%%%%%%%%%%%%%%%%%%%%%%%%%%%%%%%%%%%%%%%%%%%
%%                                          %%
%% The keywords begin here                  %%
%%                                          %%
%% Put each keyword in separate \kwd{}.     %%
%%                                          %%
%%%%%%%%%%%%%%%%%%%%%%%%%%%%%%%%%%%%%%%%%%%%%%

\begin{keyword}
\kwd{calidad}
\kwd{seguridad}
\kwd{acuerdos}
\kwd{El transporte publico}
\kwd{Corrales Carlos}
\end{keyword}

% MSC classifications codes, if any
%\begin{keyword}[class=AMS]
%\kwd[Primary ]{}
%\kwd{}
%\kwd[; secondary ]{}
%\end{keyword}

\end{abstractbox}
%
%\end{fmbox}% uncomment this for twcolumn layout

\end{frontmatter}

%%%%%%%%%%%%%%%%%%%%%%%%%%%%%%%%%%%%%%%%%%%%%%
%%                                          %%
%% The Main Body begins here                %%
%%                                          %%
%% Please refer to the instructions for     %%
%% authors on:                              %%
%% http://www.biomedcentral.com/info/authors%%
%% and include the section headings         %%
%% accordingly for your article type.       %%
%%                                          %%
%% See the Results and Discussion section   %%
%% for details on how to create sub-sections%%
%%                                          %%
%% use \cite{...} to cite references        %%
%%  \cite{koon} and                         %%
%%  \cite{oreg,khar,zvai,xjon,schn,pond}    %%
%%  \nocite{smith,marg,hunn,advi,koha,mouse}%%
%%                                          %%
%%%%%%%%%%%%%%%%%%%%%%%%%%%%%%%%%%%%%%%%%%%%%%

%%%%%%%%%%%%%%%%%%%%%%%%% start of article main body
% <put your article body there>

%%%%%%%%%%%%%%%%
%% Background %%
%%
\section*{Introducción:}
Actualmente uno de los medios de transporte más importantes y utilizados en la ciudad, con la responsabilidad de llevar a las personas a tiempo a su lugar de trabajo, de estudio,  citas importantes, de índole profesional, u informal.  
El transporte público ha ganado tanta importancia en la vida diaria de muchas personas en la ciudad y es por eso que juega un rol muy importante para la sociedad.

 %\cite{koon,oreg,khar,zvai,xjon,schn,pond,smith,marg,hunn,advi,koha,mouse}

\section*{Objetivos Generales:}

Dar a conocer las obligaciones y derechos de los choferes y usuarios del transporte público de Tijuana para crear una mejor cultura vial respecto a ellos.
%
\section*{Objetivos específicos: }
•Obtener un concepto puro de lo que es el transporte público.
\begin{itemize}
	\item Varios puntos como calidad
	\item Responsabilidades
	\item Su objetivo
\end{itemize}
•Darle un enfoque a Tijuana.
\begin{itemize}
	\item Los planes en Tijuana.
	\item El cómo se está empleando
	\item Que es lo que se quiere lograr
	\item ¿Se está logrando?
\end{itemize}

•Ver ¿Que se ha hecho en otros países o ciudades?

•Poder realizar una comparativa entre ciudades
\begin{itemize}
	\item En cual se implementa mejor y porque razón.
\end{itemize}

\section*{Cuerpo:}

\subsection{Planteamiento del problema:}

Actualmente en Tijuana el transporte público presenta ineficiencias: 
La mayoría de los usuarios presentan inconformidad con el servicio por su baja calidad y a sus tiempos no periódicos.
Los choferes presentan quejas por las injusticias en el trato  y los daños que algunos usuarios hacen a las unidades. 

\subsection{Justificación:}

Al tener una mejor cultura vial respecto al transporte público, tanto usuarios como choferes podrán defenderse de las injusticias basándose en un mejor conocimiento de los reglamentos de tránsito y vialidad.


\subsection{Marco Referencial:}

La ciudad ofrece varios tipos de transporte público. El sistema de transporte público en la ciudad de Tijuana está compuesto por autobuses, microbuses (calafias) y taxis.
La línea principal de autobuses son los "Azul y blanco" (llamados así por los colores de los autobuses, que vienen a su vez del nombre de la sociedad que tiene la concesión de estos autobuses: Transportes de Baja California Azul y Blanco, J. Magallanes, S.A. de C.V.). 

\subsection{Hipotesis:}
Si tanto choferes como usuarios conocen sus derechos y obligaciones respecto al transporte público, entonces serán capaces de defenderse de las injusticias y podrán cumplir de manera correcta con las reglas de las cuales no tienen conocimiento.
\subsection{Aspectos metodológicos:}
La investigación está basada en documentos y  una entrevista anónima a usuarios y choferes del transporte público.
Además se realizó una entrevista documentada en video al señor Margarito Ríos Martínez, chofer del transporte público; y al señor Felipe de Jesús Nevarez Enríquez, usuario. Ambas entrevistas serán mostradas en la presentación de la investigación.
\subsubsection{Estado Actual del Transporte Público  (1.1 Estadísticas)}
El servicio de transporte público de pasajeros tiene una cobertura del 91 porciento en el área urbana, y atiende al 60 porciento de la población total, con modalidades de taxis y calafias principalmente. 
El total de flota vehicular que atiende el transporte colectivo de pasajeros es de 8,389 vehículos, privilegiando la modalidad más pequeña con el 70% de unidades
\subsubsection{La demanda:}
La demanda tendencial para el área metropolitana de Tijuana contempla a los municipios de Tijuana, Tecate y Playas de Rosarito, y las zonas de crecimiento de los planes de desarrollo. Dicha demanda estima un incremento en un período de treinta años a 2, 086,865 viajes.
\subsubsection{Realidad Económica:}
En el caso particular de Tijuana, la situación económica del transporte público es preocupante, ya que las unidades son antiguas y de alto riesgo, y muchas veces están destartaladas, sin que les den mantenimiento y hasta con fallas en los frenos gran número de las unidades.
\subsubsection{Respecto a los usuarios:}
El transporte público en Tijuana es de los más caros del país, de 10 a 18 pesos por ruta, y, al mismo tiempo, de los más ineficientes. Traslada muy lento a la población y hay una gran saturación de líneas en ciertas zonas geográficas, por la disputa del mercado que se da. 
\subsubsection{Respecto a los choferes:}
Todos los choferes carecen de prestaciones como seguro social, base laboral, jubilación pagada, vacaciones o salario garantizado. En el caso de los autobuses de pasajeros, ellos obtienen su salario dependiendo de los boletos vendidos, y deben garantizar la compra diaria de gasolina, diésel y una ganancia sustanciosa al dueño de placas y camión.
\subsubsection{Impacto económico social:}
El aumento del costo del transporte sólo disminuye el poder adquisitivo de la población más necesitada, encarece la canasta básica de los tijuanenses, influye en el aumento de la inflación de la región y afecta la productividad en la ciudad. 
El incremento de las tarifas del transporte público tendrá un efecto multiplicador en el incremento de los precios de los diferentes productos, en especial de la canasta familiar en Tijuana y en todo Baja California.
\subsubsection{Problemática Actual:}
Las  políticas públicas del transporte público hasta el momento no han propiciado el correcto desarrollo del transporte masivo en Tijuana, ya que el sistema actual provoca trasbordos innecesarios y por lo mismo mayores gastos por el servicio.
Alejandro Díaz Bautista, investigador nacional del Conacyt, dijo que el carácter de público que tiene el transporte de pasajeros supone una responsabilidad del tipo de servicio que se ofrece. Implica que existe la responsabilidad de garantizar tanto la eficiencia técnica de los servicios (regularidad, uniformidad, continuidad y calidad) como la eficiencia económica con las tarifas y precios razonables para la comunidad.
\subsubsection{Calidad del transporte público:}
La frontera de Tijuana tiene el peor transporte, violento, deficiente, costoso, de alto riesgo y conflictivo de todo México. 

\subsubsection{Reglamento del Consejo Municipal del Transporte de Tijuana, Baja California}
\subsubsection{DE SU NATURALEZA Y OBJETO:}
ARTÍCULO 1.- Las disposiciones del presente reglamento son de orden público e interés social, su observancia es obligatoria en el Municipio de Tijuana Baja California y tiene por objeto el estudio, análisis, discusión y evaluación de la problemática en materia de transporte público, así como para la emisión de opiniones y recomendaciones que para su mejoramiento se estimen procedentes.

\subsubsection{DE SU INTEGRACÓN:}

Articulo 2.-El Consejo Municipal del Transporte, de conformidad con Ley del Régimen Municipal para el Estado de Baja California, la Ley General de  Transporte Público del Estado de Baja California y el Reglamento de Transporte Público para el Municipio de Tijuana Baja California, estará integrado por:
I.- Un representante de la Autoridad Municipal;
II.- Un representante designado por cada sub-comité de permisionarios y concesionarios, por cada modalidad del transporte público;
III.- Un representante ciudadano en el número igual a los anteriores, considerando invariablemente a un representante ciudadano con discapacidad a propuesta del subcomité sectorial de personas con discapacidad del COPLADEM, debiendo estos no ser funcionarios públicos, ni pertenecer a ningún sindicato. (Reforma)

ARTÍCULO 3.- El Ayuntamiento emitirá la convocatoria correspondiente estableciendo los requisitos para que se integre el Consejo Municipal del Transporte y delegara en el Presidente Municipal en conjunto con el Secretario de Gobierno Municipal la comprobación de la documentación respectiva que presenten los aspirantes, así como la aceptación de los mismos al Consejo.

ARTÍCULO 4.- En el caso de selección de los representantes designados por  cada sub-comité de permisionarios y concesionarios y para efectos de la Ley y el reglamento de la materia, se considera existen las siguientes tipos de modalidades:
1).- Transporte Colectivo de Servicio Tipo A: Primario y Express. 2).- Transporte Colectivo de Servicio Tipo B: Secundario.
3).- Transporte de Alquiler de Taxis de Ruta.
4).- Transporte de Alquiler de Taxis de Viajes Especiales. 5).- Transporte de Alquiler de Taxi Libre.
6).- Transporte Turístico.
7).- Transporte Escolar.
8).- Transporte de Personal.
9).- Transporte Público de Carga de Servicio General.
10).- Transporte Público de Carga de Servicio Especializado.
11).- Transporte Público de Carga de Servicio de Grúa y Remolque

ARTÍCULO 5.- Los ciudadanos en número igual a los  representantes mencionados en el artículo anterior de los Sub-Comités de permisionarios y concesionarios, serán elegidos (representantes de instituciones uabc, colegio de ingenieros, colegio de abogados, de maquiladoras, de turismo, etc.) de entre y por los ciudadanos que forman parte de los subcomités delegacionales del COPLADEM, quienes de preferencia deberán se usuarios continuos del servicio público de transporte.

ARTÍCULO 6.- El Consejo Municipal del Transporte, una vez elegidos sus integrantes, se instalara en un plazo no mayor de treinta días, y la sesión será convocada por el representante de la autoridad municipal y en ella se nombrara a su presidente de entre sus miembros ciudadanos, por mayoría de votos, quien durara en su encargo un año, pudiendo ser reelegido.

ARTÍCULO 7.- Los Miembros del Consejo tendrán derecho a voz y voto y desempeñaran sus funciones en forma honorífica, a excepción del Secretario Técnico del Consejo, por cada miembro propietario se nombrara un suplente.

ARTÍCULO 8.- El Presidente podrá invitar a participar en las sesiones del consejo a los integrantes de otras dependencias, entidades o agrupaciones o miembros de la sociedad cuya opinión se considere conveniente escuchar en virtud de asuntos que se traten; los invitados a participar tendrán derecho al uso de la voz, pero no tendrán derecho a voto

ARTÍCULO 9.- El Consejo contara con el personal de apoyo suficiente y necesario que al efecto autorice el Presidente Municipal, respaldándose en la asesoría de la estructura Administrativa municipal, así como Empresarial que se requiera.
\subsubsection{DE LAS FUNCIONES DEL CONSEJO:}
ARTÍCULO 10.- El Consejo Municipal del Transporte, tendrá las siguientes funciones:
I.- Conocer y opinar respecto de los programas y estudios técnicos que se realicen con el fin de adecuar y mejorar la prestación del servicio así como para determinar la necesidad real del servicio y reestructuración del mismo, atendiendo a las demandas sociales.
II.- Asesorar y emitir opiniones o recomendaciones en materia de transporte; III.- Revisar y emitir opinión del Plan Maestro de Vialidad y Transporte.
IV.- Llevar registro de los indicadores y estadísticas sobre las políticas desarrolladas en otros estados o municipios que permitan adecuar y hacer eficiente la prestación del servicio de transporte público;
V.- Opinar respecto a las medidas convenientes para hacer eficiente la prestación del servicio público de transporte;
VI.- Proponer al Presidente Municipal, políticas y programas de apoyo técnico y financiero para mejorar la prestación del servicio público de transporte así como para la entrega de permisos y concesiones;
VII.- Opinar sobre precios y tarifas que se apliquen en el servicio público de transporte previo estudio y consulta que efectúen sobre la factibilidad de su modificación;
VIII.- Recomendar el establecimiento de medidas y normas de protección a la vida y seguridad en la integridad y dignidad de los usuarios de los servicios públicos del transporte:
IX.- Elaborar conjuntamente con los prestadores del servicio público de transporte programas de capacitación para sus empleados;
X.- Promover y difundir programas para concienciar a los usuarios, respecto del servicio de transporte, tanto en el aspecto informativo para la prestación del mismo, como del cuidado responsable del equipo;
XI.- Promover los estudios de combustibles alternos para unidades de transporte; revisar y opinar modificaciones de tarifas, de conformidad a las bases que el Consejo determine;
XII.- Las demás que le sean conferidas por el Reglamento de Transporte Público para el Municipio de Tijuana B. C. con base a la Ley General de Transporte y el presente reglamento.


\subsubsection{DE LAS FUNCIONES DEL PRESIDENTE Y DEL SECRETARIO TECNICO.}
Deficiencias del transporte público en Tijuana:
Tijuana es una ciudad grande y también una ciudad muy moderna en comparación con muchas ciudades de la república, sin embargo tiene un problema y es que tiene uno de los peores servicios de transporte público del país, aparte que también es de los más caros, el servicio es muy malo y también las unidades que se están utilizando para brindar el servicio ya están muy viejas y son peligrosas ya que no se cuenta con mucha seguridad tanto para los pasajeros como para los conductores.

\subsubsection{Demanda de unidades de transporte en la ciudad:}
En Tijuana hay aproximadamente 8,398 vehículos, 70 porciento de los cuales son taxis, las unidades de transporte público de la ciudad atienden al 60porciento de la población de la misma (según datos de Plan Municipal de Desarrollo 2014-2016), tal sobresaturación de unidades (ya que la ciudad no cuenta con un sistema de transporte colectivo más amplio y eficiente, que cubra rutas más largas), provoca mayor congestionamiento vial, mayor emisión de contaminantes, escasa organización operacional y táctica, poca comodidad, altos costos, mayor cantidad de tiempo necesario en los desplazamientos y a menudo preferencia de la población por el uso de transporte privado, lo que aumenta la cantidad de vehículos en circulación y por ende la contaminación de la ciudad.

\subsubsection{El porqué de los altos precios:}
El precio del transporte público en Tijuana, según un estudio comparativo realizado por la Asociación de Transportistas Unidos del Distrito Federal, es el doble al del resto del país, lo que merma aún más la ya de por sí crítica economía de la población de escasos recursos, que son quienes utilizan mayormente dicho servicio. La modernización del transporte público tijuanense se ha visto frenada por la presión del gremio transportista por entero, son las propias empresas dueñas de las concesiones y los sindicatos de éstos, quienes unan y otra vez, por medio de paros y movilizaciones, han evitado por años la creación de un plan de transporte alterno al ya existente. 

\subsubsection{Consecuencias de la inseguridad del transporte público:}
Durante el último mes, se han suscitado en la ciudad dos accidentes de tránsito en los que estuvieron implicados taxis blanco y amarillo, pertenecientes a la ruta Tijuana–Rosarito, ambos con consecuencias fatales. El primero, acaecido el 10 de marzo, a la altura de la colonia Loma Blanca, en Tijuana, dejó un saldo de dos muertos y cuatro lesionados; el segundo, sucedido apenas el pasado 5 de abril, tuvo como resultado la muerte de cuatro personas, incluido el conductor y tres más con heridas graves.

\subsubsection{¿Cómo mejorar el transporte?}
Últimamente las TICS (Tecnologías de la información y comunicación) han tenido un auge en la sociedad moderna (Tijuana no es una excepción), este fenómeno ha creado una nueva manera de prestar servicio de transporte y esta se denomina “Uber”, pero ¿Qué es Uber?: Uber es un servicio de transporte privado que se puede pedir por medio de tu teléfono celular, este nuevo servicio tiene una mejor calidad que los transportes públicos habituales y lo mejor de todo es que también son más baratos.

\subsubsection{Comparación del transporte público en Tijuana con el del Distrito Federal:}
Medios de transporte en el DF: 
RTP
Metro
Taxi libre
Microbús o pesero

medios de transporte en Tijuana: 
Camiones
Taxis libre
Taxis de ruta

La diferencia de precios en el transporte de Tijuana y el Distrito Federal
Los precios estándares de los camiones en México son de 13-9 pesos por persona y de 5-8 pesos por un estudiante o persona de la tercera edad, para una persona de Tijuana estos precios son razonables, pero para alguien del Distrito Federal estos precios son excesivamente caros ya que en el Distrito los Microbuses cuestan entre 4.50- 5.50 pesos para una persona normal y el metro cuesta 5 pesos, ahora esto nos da una pregunta ¿por qué los precios son más caros en Tijuana? Bueno la respuesta es simple: el transporte público en Tijuana es más caros ya que son de una empresa privada y los precios debe ser más elevados para cubrir las necesidades de la empresa y además pagar a los choferes.

\subsubsection{Propuesta de Mejora:}
Si tanto choferes como usuarios conocen sus derechos y obligaciones respecto al transporte público, se puede llegar a una convivencia en la que exista un trato mutuo, en el que no haya espacio para conflictos que crean problemas entre los choferes y usuarios, que se salen de control, y solo crean situaciones que pudieron haber sido evitadas.

\subsubsection{choferes:}
Son quienes más deberían tenerlos en cuenta ya que se dedican a este servicio, y si llegaran a hacer las cosas de la manera no debida entonces quienes saldrían perdiendo en su totalidad serian ellos, ya que viven de esto.
Si se hacen valer como deben ser sus derechos y obligaciones, evitan cualquier posibilidad de perder su empleo, y además pueden llegar a hacer más popular sus servicios, llevando esto a que tengan más clientes.

\subsubsection{usuarios:}
Tienen que tener un poco más de cuidado, ya que ellos pueden ser quienes lleguen a provocar al chofer a tomar una mala decisión y esto vendría afectando no solo a los presentes en el momento, sino que se tomarían medidas para evitar cualquier otro evento parecido, y si se llega a ser muy rigurosa, se aplicaría lo decidido no solo con el chofer que ocurrió, sino con otros que sean socios.

\section*{Conclusiones:}
Conocer nuestros derechos es importante en todos los ámbitos de nuestra vida, aunque del transporte público se trate, sin embargo también nuestras obligaciones deben ser cumplidas por nosotros. Si todos conociéramos los reglamentos como es debido, podríamos contribuir a ayudar el mejoramiento del transporte. Hacer frente a las empresas y exigir un servicio eficiente y justo nos ayudará a progresar como sociedad, de lo contrario continuaremos cometiendo los mismos errores creyendo que no podemos hacer nada al respecto. 



%%%%%%%%%%%%%%%%%%%%%%%%%%%%%%%%%%%%%%%%%%%%%%
%%                                          %%
%% Backmatter begins here                   %%
%%                                          %%
%%%%%%%%%%%%%%%%%%%%%%%%%%%%%%%%%%%%%%%%%%%%%%

\begin{backmatter}
%%%%%%%%%%%%%%%%%%%%%%%%%%%%%%%%%%%%%%%%%%%%%%%%%%%%%%%%%%%%%
%%                  The Bibliography                       %%
%%                                                         %%
%%  Bmc_mathpys.bst  will be used to                       %%
%%  create a .BBL file for submission.                     %%
%%  After submission of the .TEX file,                     %%
%%  you will be prompted to submit your .BBL file.         %%
%%                                                         %%
%%                                                         %%
%%  Note that the displayed Bibliography will not          %%
%%  necessarily be rendered by Latex exactly as specified  %%
%%  in the online Instructions for Authors.                %%
%%                                                         %%
%%%%%%%%%%%%%%%%%%%%%%%%%%%%%%%%%%%%%%%%%%%%%%%%%%%%%%%%%%%%%

% if your bibliography is in bibtex format, use those commands:
\bibliographystyle{bmc-mathphys} % Style BST file (bmc-mathphys, vancouver, spbasic).
\bibliography{bmc_article}      % Bibliography file (usually '*.bib' )
% for author-year bibliography (bmc-mathphys or spbasic)
% a) write to bib file (bmc-mathphys only)
% @settings{label, options="nameyear"}
% b) uncomment next line
%\nocite{label}

% or include bibliography directly:
% \begin{thebibliography}
% \bibitem{b1}
% \end{thebibliography}

%%%%%%%%%%%%%%%%%%%%%%%%%%%%%%%%%%%
%%                               %%
%% Figures                       %%
%%                               %%
%% NB: this is for captions and  %%
%% Titles. All graphics must be  %%
%% submitted separately and NOT  %%
%% included in the Tex document  %%
%%                               %%
%%%%%%%%%%%%%%%%%%%%%%%%%%%%%%%%%%%

%%
%% Do not use \listoffigures as most will included as separate files

\section*{Figures}
  \begin{figure}[h!]
  \caption{\csentence{Sample figure title.}
      A short description of the figure content
      should go here.}
      \end{figure}

\begin{figure}[h!]
  \caption{\csentence{Sample figure title.}
      Figure legend text.}
      \end{figure}

%%%%%%%%%%%%%%%%%%%%%%%%%%%%%%%%%%%
%%                               %%
%% Tables                        %%
%%                               %%
%%%%%%%%%%%%%%%%%%%%%%%%%%%%%%%%%%%

%% Use of \listoftables is discouraged.
%%
\section*{Tables}
\begin{table}[h!]
\caption{Sample table title. This is where the description of the table should go.}
      \begin{tabular}{cccc}
        \hline
           & B1  &B2   & B3\\ \hline
        A1 & 0.1 & 0.2 & 0.3\\
        A2 & ... & ..  & .\\
        A3 & ..  & .   & .\\ \hline
      \end{tabular}
\end{table}

%%%%%%%%%%%%%%%%%%%%%%%%%%%%%%%%%%%
%%                               %%
%% Additional Files              %%
%%                               %%
%%%%%%%%%%%%%%%%%%%%%%%%%%%%%%%%%%%

\section*{Additional Files}
  \subsection*{Additional file 1 --- Sample additional file title}
    Additional file descriptions text (including details of how to
    view the file, if it is in a non-standard format or the file extension).  This might
    refer to a multi-page table or a figure.

  \subsection*{Additional file 2 --- Sample additional file title}
    Additional file descriptions text.


\end{backmatter}
\end{document}
